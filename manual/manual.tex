\documentclass[11pt]{amsart}
\usepackage[margin=.5in]{geometry}
\geometry{letterpaper}
\usepackage{graphicx}
\usepackage{amssymb}
\usepackage{epstopdf}
\DeclareGraphicsRule{.tif}{png}{.png}{`convert #1 `dirname #1`/`basename #1 .tif`.png}
\usepackage{physics}
\usepackage[dvipsnames]{xcolor}
\usepackage{listings}
\usepackage{hyperref}
\usepackage{multicol}
\definecolor{codegreen}{rgb}{0,0.6,0}
\definecolor{codegray}{rgb}{0.5,0.5,0.5}
\definecolor{codepurple}{rgb}{0.58,0,0.82}
\definecolor{backcolour}{rgb}{0.95,0.95,0.92}
\lstdefinestyle{mystyle}{
    backgroundcolor=\color{backcolour},
    commentstyle=\color{codegreen},
    keywordstyle=\color{magenta},
    numberstyle=\tiny\color{codegray},
    stringstyle=\color{codepurple},
    basicstyle=\ttfamily\footnotesize,
    breakatwhitespace=false,
    breaklines=true,
    captionpos=b,
    keepspaces=true,
    numbers=left,
    numbersep=5pt,
    showspaces=false,
    showstringspaces=false,
    showtabs=false,
    tabsize=2
}
\lstset{style=mystyle}

\title{A FORTRAN Program for Experimental WIMP Analysis}
\author{Oliver Gorton, Changfeng Jiao and Calvin Johnson}
%\date{}                                           % Activate to display a given date or no date

\begin{document}
\maketitle

{
\centering

Model-Independent WIMP Scattering Responses and Event Rates

}

\tableofcontents

This is the manual for the FORTRAN version of the model-independent WIMP scattering response and event rate code, which was originally written in Mathematica and described in \href{http://arxiv.org/abs/1308.6288v1}{arXiv:1308.6288}.

\section{Primary functions}
\subsection{Differential event rate}
\begin{equation}\label{ER}
\begin{split}
	\frac{dR_D}{dE_R} = \frac{dR_D}{d\vec{q}^2}(q)
	 = N_T n_\chi \int_{v_{min}}^\infty \frac{d\sigma(v,q)}{d\vec{q}^2} \tilde{f}(\vec{v})vd^3v
\end{split}
\end{equation}
where $q$ is the WIMP-nucleon momentum transfer, $N_T$ is the number of target nuclei, $n_\chi = \rho_\chi/m_\chi$ is the local dark matter density, $\sigma$ is the WIMP-nucleon cross section, and $\tilde{f}$ is dark matter velocity distribution in the lab-frame. $\tilde{f}(\vec{v})$ is obtained by boosting the Galactic-frame distribution $f(\vec{v})$, 

\begin{equation}\label{boost}
\begin{split}
	\tilde{f}(\vec{v}) = f(\vec{v} + \vec{v}_{earth}),
\end{split}
\end{equation}
where $\vec{v}_{earth}$ is the velocity of the earth in the galactic rest frame. The simplest model is a three-dimensional Maxwell distribution:

\begin{equation}
\begin{split}
	f(\vec{v}) \propto e^{-\vec{v}^2/v_0^2},
\end{split}
\end{equation}
where $v_0$ is some scaling factor (typically taken to be around $220\ km/s$).

In order to evaluate the integral in (\ref{ER}), we make the conversion to spherical coordinates, and take special care to deal with the velocity boost in (\ref{boost}). Assuming a $1/v^2$ velocity dependence of the cross section term (see section \ref{crosssection}), we need to evaluate an integral of the form

\begin{equation}
\begin{split}
	I = \int_{v_{min}}^{v_{max}} d^3v \frac{f(\vec{v} + \vec{v}_{earth})}{v} =
    	\int_{v_{min}}^{v_{max}} d^3v \frac{1}{v} e^{-(\vec{v}+\vec{v}_{earth})^2/v_0^2}
\end{split}
\end{equation}
Noting that $(\vec{v}+\vec{v}_{earth})^2 = \vec{v}^2 + \vec{v}^2_{earth} + 2vv_{earth}\cos(\theta)$, with $|\vec{v}|\equiv v$ and $\theta$ defining the angle between the two vectors, it's convenient to make the substitution $d^3v = d\phi d(\cos \theta) v^2 dv$:

\begin{equation}
\begin{split}
	I &=  \int_0^{2\pi} d\phi \int_{v_{min}}^{v_{max}} dv \int_{-1}^1 d(\cos \theta) e^{-2vv_{earth}\cos\theta/v_0^2} v^2 \frac{1}{v} e^{-(\vec{v}^2+\vec{v}^2_{earth})/v_0^2}\\
	&= 2\pi \int_{v_{min}}^{v_{max}} dv v e^{-(\vec{v}^2+\vec{v}^2_{earth})/v_0^2} \left(-\frac{v_0^2}{2vv_{earth}} e^{-2vv_{earth}\cos\theta/v_0^2}\right)_{-1}^1\\
	&= \frac{\pi v_0^2}{v_{earth} }\int_{v_{min}}^{v_{max}} dv e^{-(\vec{v}^2+\vec{v}^2_{earth})/v_0^2} 
		\left(- e^{-2vv_{earth}/v_0^2} + e^{+2vv_{earth}/v_0^2}\right)\\
	&= \frac{\pi v_0^2}{v_{earth} }\int_{v_{min}}^{v_{max}} dv 
		\left(- e^{(v+v_{earth})^2/v_0^2} + e^{(v-v_{earth})^2/v_0^2}\right)\\
	&= \frac{\pi v_0^2}{v_{earth} }\int_{v_{min}}^{v_{max}} dv 
		\left( g(v-v_{earth}) - g(v+v_{earth}) \right)
\end{split}
\end{equation}
where in the last equality, we have defined a one-dimensional Gaussian form
\begin{equation}
\begin{split}
	g(v) \propto e^{-v^2/v_0^2}.
\end{split}
\end{equation}

The final expression for $I$ can be trivially generalized to other spherically-symmetric velocity-dependent forms of the differential cross section. What's important is the reduction of the velocity-boosted $d^3v$ integral to a radial integral which can be carried out with one-dimensional quadrature:
\begin{equation}\label{integral}
\begin{split}
\int_{v_{min}}^{v_{max}} d^3v \sigma(v) e^{-(\vec{v}+\vec{v}_{earth})^2/v_0^2} 
	= \frac{\pi v_0^2}{v_{earth} }\int_{v_{min}}^{v_{max}} dv \sigma(v) v^2\left( g(v-v_{earth}) - g(v+v_{earth}) \right).
\end{split}
\end{equation}
The FORTRAN code uses equation (\ref{integral}) to evaluate the event rate integral in equation (\ref{ER}) with quadrature. Analytic solutions of  (\ref{integral}) exist in the form of error functions; we use the above form since it makes easy to later modify the velocity distribution (as long as it remains spherically symmetric). For example, adding a velocity cut-off is as easy as changing the limit on the quadrature, with no need to write a whole new subroutine for the analytic forms found in the Mathematica script.

\subsection{Differential cross section}\label{crosssection}
\begin{equation}
\begin{split}
\frac{d\sigma(v,E_R)}{dE_R} = 2m_T \frac{d\sigma(v)(v,\vec{q}^2)}{d\vec{q}^2} = 2m_T\frac{1}{4\pi v^2}T(v,q),
\end{split}
\end{equation}
Where $v$ is the velocity of the dark matter particles in the lab-frame, $q$ is the momentum transfer of the scattering event, $m_T$ is the mass of the target nucleus, and $T(v,q)$ is the transition or scattering probability. We can see here that the differential cross section has an explicit $1/v^2$ dependence, independent of any velocity dependence of $T(v,q)$.


\subsection{Transition probability / Scattering probability}

The scattering probability is

\begin{equation}
\begin{split}
T(v,q) &= \frac{1}{2j_\chi+1}\frac{1}{2j_T+1}\sum_{spins}|\mathcal{M}(v,q)|^2 
\end{split}
\end{equation}
where $j_\chi$ is the spin of the WIMP, $j_T$ is the spin angular momentum of the target nucleus, and $\mathcal{M}$ Galilean invariant amplitude:
\begin{align}
	T(v,q) = \frac{4\pi}{2j_T+1}\frac{1}{(4m_\chi)^2}
		\sum_{\tau=0}^1\sum_{\tau'=0}^1\sum_{i=1}^8 R_i^{\tau\tau'}(v^2,q^2)
		W_i^{\tau\tau'}(q)
%		\left < \mathcal{O}_{j_T,\tau}^{(i)}(y)\right >
%		\left < \mathcal{O'}_{j_T,\tau'}^{(i)}(y)\right >
\end{align}
where $m_\chi$ is the mass of the dark matter particle and $\tau$ is an index used to sum over isospin couplings. The coefficients $R_i^{\tau,\tau'}$ are dark matter particle response functions, to be define in another section. The operators $W_i^{\tau\tau'}(q)$ are nuclear response functions, which are sums over matrix elements of nuclear operators constructed from Bessel spherical harmonics and vector spherical harmonics.
\subsection{Dark matter response functions}
There are 8 dark matter response functions which group 15 operator coefficients $c_i^\tau$ according the pair of nuclear response functions which they multiply.

\begin{align}
R_{M}^{\tau\tau'}(v,q) &= 
\frac{1}{4}cl(j_\chi) \left(
	(v^2-(\frac{q}{2\mu_t})^2)(c_5^{\tau}c_5^{\tau'}q^2 + c_8^{\tau}c_8^{\tau'})
	+ c_{11}^{\tau}c_{11}^{\tau'}q^2
\right)\\
& + (c_1^{\tau} + c_2^{\tau}(v^2-(\frac{q}{2\mu_t})^2) )(c_1^{\tau'} + c_2^{\tau'}(v^2-(\frac{q}{2\mu_t})^2) )
\\
R_{\Sigma''}^{\tau\tau'}(v,q) &= 
\frac{1}{16}cl(j_\chi) \left(
	c_6^{\tau}c_6^{\tau'}q^4 + (c_{13}^{\tau}c_{13}^{\tau'}q^2 +c_{12}^{\tau} c_{12}^{\tau'} )
	(v^2-q^2/(2\mu_T)^2) + 2c_4^\tau c_6^{\tau'} q^2 + c_4^\tau c_4^{\tau '}
\right )
+ \frac{1}{4} c_{10}^\tau c_{10}^{\tau '} q^2
\\
R_{\Sigma'}^{\tau\tau'}(v,q) &= 
\frac{1}{32} cl(j_\chi) \left (
	2c_{9}^{\tau}c_{9}^{\tau'}q^2
	+ (
		c_{15}^{\tau}c_{15}^{\tau'}q^4
		+ c_{14}^{\tau}c_{14}^{\tau'}q^2
		-2 c_{12}^{\tau}c_{15}^{\tau'} q^2
		+c_{12}^{\tau}c_{12}^{\tau'}
	) (v^2-q^2/(2\mu_T)^2)
	+2 c_{4}^{\tau}c_{4}^{\tau'}
\right)\\
&+\frac{1}{8} \left(
	c_{3}^{\tau}c_3^{\tau'}q^2
	+ c_{7}^{\tau}c_{7}^{\tau'}
\right) (v^2-q^2/(2\mu_T)^2)
\\
R_{\Phi''}^{\tau\tau'}(v,q) &= 
\frac{q^2}{(4m_N)^2}cl(j_\chi) \left(
	c_{12}^\tau - c_{15}^{\tau} q^2
\right ) \left(
	c_{12}^{\tau '}-c_{15}^{\tau '}q^2
\right )
+ \frac{q^2}{(4m_N)^2}q^2c_3^\tau c_3^{\tau'}
\\
R_{\tilde{\Phi}'}^{\tau\tau'}(v,q) &= 
\frac{q^2}{(4m_N)^2}cl(j_\chi) \left(
	c_{12}^\tau c_{12}^{\tau'} q^2
	+ c_{12}^\tau c_{12}^{\tau'} 
\right)
\\
R_{\Delta}^{\tau\tau'}(v,q) &= 
\frac{q^2}{(2m_N)^2}cl(j_\chi) \left(
	c_{5}^{\tau}c_{5}^{\tau'}q^2
	+ c_{8}^{\tau}c_{8}^{\tau'}
\right)
+2\frac{q^2}{m_N^2} c_{2}^{\tau}c_{2}^{\tau'}
(v^2-q^2/(2\mu_T)^2)
\\
R_{\Delta \Sigma'}^{\tau\tau'}(v,q) &= 
\frac{q^2}{(2m_N)^2}cl(j_\chi) \left(
c_{4}^{\tau}c_{5}^{\tau'} - c_{8}^{\tau}c_{9}^{\tau'}
\right)
- \frac{q^2}{m_N}c_{2}^{\tau}c_{3}^{\tau'} (v^2-q^2/(2\mu_T)^2)
\\
R_{\Phi''M}^{\tau\tau'}(v,q) &= 
\frac{q^2}{4m_N} cl(j_\chi) c_{11}^{\tau} 
\left(
c_{12}^{\tau'} - c_{12}^{\tau'} q^2
\right)
+ \frac{q^2}{m_N} c_{3}^{\tau'} \left(
	c_{1}^{\tau} + c_{2}^{\tau} (v^2-q^2/(2\mu_T)^2)
\right)
\end{align}
As a shorthand we have introduced the notation 
\begin{align}
	cl(j) = 4j(j+1)/3.
\end{align}


\subsection{Nuclear response functions}
There are eight nuclear response functions $W_i^{\tau\tau'}(y)$ considered here. The unit-less variable $y$ is defined 
\begin{align}
 y = \left ( \frac{qb}{2} \right) ^2,
 \end{align}
 in terms of the harmonic oscillator size parameter $b$, which has a default value of 
 \begin{align}
b^2 = 41.467/(45A^{-1./3} - 25A^{-2/3})\ fm^2.
 \end{align}

 
\begin{align}
W_M^{\tau\tau'}(y) &= \sum_{even\ J} 
\bra{j_T}M_{J\tau}(y)\ket{j_T}
\bra{j_T}M_{J\tau'}(y)\ket{j_T}
\\
W_{\Sigma''}^{\tau\tau'}(y) &= \sum_{odd\ J} 
\bra{j_T}{\Sigma''}_{J\tau}(y)\ket{j_T}
\bra{j_T}{\Sigma''}_{J\tau'}(y)\ket{j_T}
\\
W_{\Sigma'}^{\tau\tau'}(y) &= \sum_{odd\ J} 
\bra{j_T}{\Sigma'}_{J\tau}(y)\ket{j_T}
\bra{j_T}{\Sigma'}_{J\tau'}(y)\ket{j_T}
\\
W_{\Phi''}^{\tau\tau'}(y) &= \sum_{even\ J} 
\bra{j_T}{\Phi''}_{J\tau}(y)\ket{j_T}
\bra{j_T}{\Phi''}_{J\tau'}(y)\ket{j_T}
\\
W_{\tilde{\Phi}'}^{\tau\tau'}(y) &= \sum_{even\ J} 
\bra{j_T}{\tilde{\Phi}'}_{J\tau}(y)\ket{j_T}
\bra{j_T}{\tilde{\Phi}'}_{J\tau'}(y)\ket{j_T}
\\
W_{\Delta}^{\tau\tau'}(y) &= \sum_{odd\ J} 
\bra{j_T}{\Delta}_{J\tau}(y)\ket{j_T}
\bra{j_T}{\Delta}_{J\tau'}(y)\ket{j_T}
\\
W_{\Delta\Sigma'}^{\tau\tau'}(y) &= \sum_{odd\ J} 
\bra{j_T}{\Delta}_{J\tau}(y)\ket{j_T}
\bra{j_T}{\Sigma'}_{J\tau'}(y)\ket{j_T}
\\
W_{\Phi''M}^{\tau\tau'}(y) &= \sum_{even\ J} 
\bra{j_T}{\Phi''}_{J\tau}(y)\ket{j_T}
\bra{j_T}{M}_{J\tau'}(y)\ket{j_T}
\end{align}


\subsection{Nuclear operators and their matrix elements}
of which there are six, are nuclear operators constructed from Bessel spherical harmonics and vector spherical harmonics, and are evaluated here on the ground state of the target nucleus.

\section{Inputs and control file usage}
This code has a twofold user-interface scheme. This first is through the interactive command-line interface. The program will prompt the user for the minimum necessary inputs to run a calculation with default parameter values. In the case of a differential event rate calculation, these would be the momentum transfer $q$, information about the target nucleus including the density matrices, and finally the coefficient of the WIMP-nucleus interaction. The coefficients are read in from the second user-interface scheme: the control file. 

The simplest and necessary use case of the control file is to read in the coefficients of the WIMP-nucleon interaction. Each coefficient is stored on its one line in a file which ends in .control, with 4 entries each: the keyword "coefnonrel", a 0 or 1 indicating proton or neutron coupling, the index of the coefficient to set, and finally the value of the coefficient. These entries can be separated by tabs or spaces. For example, to set $c_5^{(n)}=3.1$, the corresponding line in the control file would be:
\begin{verbatim}
coefnonrel    1    5     3.1
\end{verbatim}

The control file also serves a more general but optional function: to set any parameter in the program to a custom value. To use this functionality is simple: simply add an entry to the control file with two values: the first should be the keyword for the parameter (see section \ref{keywords}) and the second should be the value to set that parameter to. For example, to set the velocity of the earth in the galactic frame to $240\ km/s$, you should add the line:
\begin{verbatim}
vearth  240.0
\end{verbatim}

\section{Keywords}\label{keywords}

\section{Example: si28 differential scattering rate}
\small
\begin{verbatim}
 %%%%%%%%%%%%%%%%%%%%%%%%%%%%%%%%%%%%%%%%%%%%%%%%%%%%%%%%%%%%%%%%%%%%%%%%%%%%%%%
 Welcome to our FORTAN 90 implementation of the WIMP-nucleon scattering code.
 Based on the Mathematica script described in arXiv:1308.6288 (2003).
   VERSION 1.1 UPDATED: 2020.05.23 @ SDSU
   Dev. contact: cjohnson@sdsu.edu
                  ogorton@sdsu.edu
 %%%%%%%%%%%%%%%%%%%%%%%%%%%%%%%%%%%%%%%%%%%%%%%%%%%%%%%%%%%%%%%%%%%%%%%%%%%%%%%
 
 Select an option:
 [1] Differential scattering rate per detector per unit recoil energy.
 [2] Scattering probability.
 [3] Differential cross section per recoil energy.
 [4] (Future feature) Total cross section.
 [5] (Future feature) Total scattering rate per detector.
 [6] (Future featuer) 
1
 Enter q, the three-momentum transfer of the scattering reaction:
1
  
 Enter the neutron number 
14
  
 Enter the proton number 
14
 Setting default parameter values.
  
 Enter name of control file (.control):
../si28
 %%%%%%%%%%%%%%%%%%%%%%%%%%%%%%%%%%%%%%%%%%%%%%%%%%%%%%%%%%%%%%%%%%%%%%%%%%%%%%%
 Reading control file.
 Possible keywords:
 coefnonrel          
 vearth              
 dmdens              
 quadtype            
 intpoints           
 gev                 
 femtometer          
 dmmass              
 vescape             
 ntarget             
 weakmscale          
 vscale              
 mnucleon            
 dmspin              
 
 # Coefficient matrix
 # Ommitted values ar
 # c_i^t             
 # i = 1,...,16      
 # t = 0 protons, 1 n
 #control name	 t	i	c
 Set non-relativistic coefficient: op           1 p/n           1 c   3.1000000000000001     
 Set non-relativistic coefficient: op           3 p/n           1 c   3.1000000000000001     
 Set non-relativistic coefficient: op           4 p/n           1 c   3.1000000000000001     
 Set non-relativistic coefficient: op           5 p/n           1 c   3.1000000000000001     
 Set non-relativistic coefficient: op           6 p/n           1 c   3.1000000000000001     
 Set non-relativistic coefficient: op           7 p/n           1 c   3.1000000000000001     
 Set non-relativistic coefficient: op           8 p/n           1 c   3.1000000000000001     
 Invalid keyword "fakekeyword". Ignoring.
 vearth: Set velocity of earth in galactic frame set to   232.00000000000000     
 dmdens: Set local dark matter density to   1.0000000000000000     
 intpoints: Set number of integral lattice points to        1000
 End of control file.
 
  Enter shell-model space file name (.sps)
../sd
 Shell-Model space file name ../sd          
           1           0           3           2
           2           0           5           2
           3           1           1           0
           4           0           1           1
           5           0           3           1
           6           0           1           0
  
  Enter name of one-body density file (.res) 
../si28w
           6
 Fill core? [y/n]
y
 Filling core orbitals.
 Printing density matrix.
 # spo =           6
 J=           0
           1           1  0.475710005       0.00000000    
           2           2   2.66933990       0.00000000    
           3           3  0.703809977       0.00000000    
           4           4   2.00000000       2.00000000    
           5           5   2.82842708       2.82842708    
           6           6   2.00000000       2.00000000    
 J=           1
 J=           2
 J=           3
 J=           4
 J=           5
 J=           6
 J=           7
 J=           8
 J=           9
 J=          10
 b[dimless]=   1.8506094217300415     
 b[fm]=   9.3783331984848814     
 y=   21.988283395450917     
 mN  0.93827199935913086     
 jchi  0.50000000000000000     
 mchi   50.000000000000000     
 Jiso=           0
 Tiso=           0
 Mtiso=           0
 ap,an          14          14
 Miso=   28.000000000000000     
 muT=   17.222406817915434     
 q=   1.0000000000000000     
 vdist_min =    2.9031946886766143E-002
 vdist_max =    2640.0000000000000     
 nt   1.0000000000000000     
 rhochi   1.0000000000000000     
 mchi   50.000000000000000     
 v0=   220.00000000000000     
 ve=   232.00000000000000     
 Integral lattice size =         1000
 %%%%%%%%%%%%%%%%%%%%%%%%%%%%%%%%%%%%%%%%%%%%%%%%%%%%%%%%%%%%%%%%%%%%%%%%%%%%%%%
 Event rate =    1.3425254633130306E-021
 %%%%%%%%%%%%%%%%%%%%%%%%%%%%%%%%%%%%%%%%%%%%%%%%%%%%%%%%%%%%%%%%%%%%%%%%%%%%%%%
\end{verbatim}

\end{document}  


\begin{equation}
\begin{split}
\end{split}
\end{equation}
