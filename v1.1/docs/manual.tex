\documentclass[11pt]{amsart}
\usepackage[margin=.5in]{geometry}
\geometry{letterpaper}
\usepackage{graphicx}
\usepackage{amssymb}
\usepackage{epstopdf}
\DeclareGraphicsRule{.tif}{png}{.png}{`convert #1 `dirname #1`/`basename #1 .tif`.png}
\usepackage{physics}
\usepackage[dvipsnames]{xcolor}
\usepackage{listings}
\usepackage{hyperref}
\usepackage{multicol}
\definecolor{codegreen}{rgb}{0,0.6,0}
\definecolor{codegray}{rgb}{0.5,0.5,0.5}
\definecolor{codepurple}{rgb}{0.58,0,0.82}
\definecolor{backcolour}{rgb}{0.95,0.95,0.92}
\lstdefinestyle{mystyle}{
    backgroundcolor=\color{backcolour},
    commentstyle=\color{codegreen},
    keywordstyle=\color{magenta},
    numberstyle=\tiny\color{codegray},
    stringstyle=\color{codepurple},
    basicstyle=\ttfamily\footnotesize,
    breakatwhitespace=false,
    breaklines=true,
    captionpos=b,
    keepspaces=true,
    numbers=left,
    numbersep=5pt,
    showspaces=false,
    showstringspaces=false,
    showtabs=false,
    tabsize=2
}
\lstset{style=mystyle}

\title{Model-Independent WIMP Scattering Responses and Event Rates:\\ 
A FORTRAN Program for Experimental Analysis}
\author{Oliver Gorton and Calvin Johnson}
%\date{}                                           % Activate to display a given date or no date

\begin{document}
\maketitle

\tableofcontents

This is the manual for the FORTRAN version of the model-independent WIMP scattering response and event rate code, which was originally written in Mathematica and described in \href{http://arxiv.org/abs/1308.6288v1}{arXiv:1308.6288}.

\section{Theory}

\section{Primary functions}

\subsection{Differential event rate}
\begin{equation}\label{ER}
\begin{split}
	\frac{dR_D}{dE_R} = \frac{dR_D}{d\vec{q}^2}(q)
	 = N_T n_\chi \int_{v_{min}}^\infty \frac{d\sigma(v,q)}{d\vec{q}^2} \tilde{f}(\vec{v})vd^3v
\end{split}
\end{equation}
where $q$ is the WIMP-nucleon momentum transfer, $N_T$ is the number of target nuclei, $n_\chi = \rho_\chi/m_\chi$ is the local dark matter density, $\sigma$ is the WIMP-nucleon cross section, and $\tilde{f}$ is dark matter velocity distribution in the lab-frame. $\tilde{f}(\vec{v})$ is obtained by boosting the Galactic-frame distribution $f(\vec{v})$, 

\begin{equation}\label{boost}
\begin{split}
	\tilde{f}(\vec{v}) = f(\vec{v} + \vec{v}_{earth}),
\end{split}
\end{equation}
where $\vec{v}_{earth}$ is the velocity of the earth in the galactic rest frame. The simplest model is a three-dimensional Maxwell distribution:

\begin{equation}
\begin{split}
	f(\vec{v}) \propto e^{-\vec{v}^2/v_0^2},
\end{split}
\end{equation}
where $v_0$ is some scaling factor (typically taken to be around $220\ km/s$).

In order to evaluate the integral in (\ref{ER}), we make the conversion to spherical coordinates, and take special care to deal with the velocity boost in (\ref{boost}). Assuming a $1/v^2$ velocity dependence of the cross section term (see section \ref{crosssection}), we need to evaluate an integral of the form

\begin{equation}
\begin{split}
	I = \int_{v_{min}}^{v_{max}} d^3v \frac{f(\vec{v} + \vec{v}_{earth})}{v} =
    	\int_{v_{min}}^{v_{max}} d^3v \frac{1}{v} e^{-(\vec{v}+\vec{v}_{earth})^2/v_0^2}
\end{split}
\end{equation}
Noting that $(\vec{v}+\vec{v}_{earth})^2 = \vec{v}^2 + \vec{v}^2_{earth} + 2vv_{earth}\cos(\theta)$, with $|\vec{v}|\equiv v$ and $\theta$ defining the angle between the two vectors, it's convenient to make the substitution $d^3v = d\phi d(\cos \theta) v^2 dv$:

\begin{equation}
\begin{split}
	I &=  \int_0^{2\pi} d\phi \int_{v_{min}}^{v_{max}} dv \int_{-1}^1 d(\cos \theta) e^{-2vv_{earth}\cos\theta/v_0^2} v^2 \frac{1}{v} e^{-(\vec{v}^2+\vec{v}^2_{earth})/v_0^2}\\
	&= 2\pi \int_{v_{min}}^{v_{max}} dv v e^{-(\vec{v}^2+\vec{v}^2_{earth})/v_0^2} \left(-\frac{v_0^2}{2vv_{earth}} e^{-2vv_{earth}\cos\theta/v_0^2}\right)_{-1}^1\\
	&= \frac{\pi v_0^2}{v_{earth} }\int_{v_{min}}^{v_{max}} dv e^{-(\vec{v}^2+\vec{v}^2_{earth})/v_0^2} 
		\left(- e^{-2vv_{earth}/v_0^2} + e^{+2vv_{earth}/v_0^2}\right)\\
	&= \frac{\pi v_0^2}{v_{earth} }\int_{v_{min}}^{v_{max}} dv 
		\left(- e^{(v+v_{earth})^2/v_0^2} + e^{(v-v_{earth})^2/v_0^2}\right)\\
	&= \frac{\pi v_0^2}{v_{earth} }\int_{v_{min}}^{v_{max}} dv 
		\left( g(v-v_{earth}) - g(v+v_{earth}) \right)
\end{split}
\end{equation}
where in the last equality, we have defined a one-dimensional Gaussian form
\begin{equation}
\begin{split}
	g(v) \propto e^{-v^2/v_0^2}.
\end{split}
\end{equation}

The final expression for $I$ can be trivially generalized to other spherically-symmetric velocity-dependent forms of the differential cross section. What's important is the reduction of the velocity-boosted $d^3v$ integral to a radial integral which can be carried out with one-dimensional quadrature:
\begin{equation}\label{integral}
\begin{split}
\int_{v_{min}}^{v_{max}} d^3v \sigma(v) e^{-(\vec{v}+\vec{v}_{earth})^2/v_0^2} 
	= \frac{\pi v_0^2}{v_{earth} }\int_{v_{min}}^{v_{max}} dv \sigma(v) v^2\left( g(v-v_{earth}) - g(v+v_{earth}) \right).
\end{split}
\end{equation}
The FORTRAN code uses equation (\ref{integral}) to evaluate the event rate integral in equation (\ref{ER}). Of course, analytic solutions of  (\ref{integral}) exist in the form of error functions, but the above form allows for evaluation with quadrature, making it easy to later modify the velocity distribution (as long as it remains spherically symmetric). For example, adding a velocity cut-off is as easy as changing the limit on the quadrature, with no need to write a whole new subroutine for the analytic forms found in the Mathematica script.

\subsection{Differential cross section}\label{crosssection}
\begin{equation}
\begin{split}
\frac{d\sigma(v,E_R)}{dE_R} = 2m_T \frac{d\sigma(v)(v,\vec{q}^2)}{d\vec{q}^2} = 2m_T\frac{1}{4\pi v^2}T(v,q),
\end{split}
\end{equation}
Where $v$ is the velocity of the dark matter particles in the lab-frame, $q$ is the momentum transfer of the scattering event, $m_T$ is the mass of the target nucleus, and $T(v,q)$ is the transition or scattering probability. We can see here that the differential cross section has an explicit $1/v^2$ dependence, independent of any velocity dependence of $T(v,q)$.


\subsection{Transition probability / Scattering probability}


\section{Inputs and control file usage}




\end{document}  


\begin{equation}
\begin{split}
\end{split}
\end{equation}