\documentclass[
14pt, % Main document font size
a4paper, % Paper type, use 'letterpaper' for US Letter paper
oneside, % One page layout (no page indentation)
%twoside, % Two page layout (page indentation for binding and different headers)
headinclude,footinclude, % Extra spacing for the header and footer
BCOR5mm, % Binding correction
]{scrartcl}
\input{structure.tex}

\usepackage{geometry}
\geometry{margin=1.in}
\usepackage{graphicx}
\usepackage{amssymb}
\usepackage{epstopdf}
\DeclareGraphicsRule{.tif}{png}{.png}{`convert #1 `dirname #1`/`basename #1 .tif`.png}
\usepackage{physics}
\usepackage[dvipsnames]{xcolor}
\usepackage{listings}
\usepackage{hyperref}
\usepackage{multicol}
\usepackage{longtable}
\usepackage{breqn}

%New colors defined below
\definecolor{codegreen}{rgb}{0,0.6,0}
\definecolor{codegray}{rgb}{0.5,0.5,0.5}
\definecolor{codepurple}{rgb}{0.58,0,0.82}
\definecolor{backcolour}{rgb}{0.95,0.95,0.92}

%Code listing style named "mystyle"
\lstdefinestyle{mystyle}{
  backgroundcolor=\color{backcolour},   
  commentstyle=\color{codegreen},
  keywordstyle=\color{magenta},
  numberstyle=\tiny\color{codegray},
  stringstyle=\color{codepurple},
  basicstyle=\ttfamily\small,
  breakatwhitespace=false,
  breaklines=true,
  captionpos=b,
  keepspaces=true,
  numbers=left,
  numbersep=5pt,
  showspaces=false,
  showstringspaces=false,
  showtabs=false,
  tabsize=2
}

\lstset{style=mystyle}

\title{DMFortFactor: A FORTRAN Program for Experimental WIMP Analysis}
\author{Oliver Gorton, Changfeng Jiao and Calvin Johnson}

\begin{document}

\fontfamily{cmss}\selectfont
\setlength{\parindent}{0em}
\setlength{\parskip}{1em}

\maketitle

{

\centering

Model-Independent WIMP Scattering Responses and Event Rates

}

\tableofcontents


\clearpage

\section{Preamble}
This is the manual for the FORTRAN version of the model- independent WIMP 
scattering response and event rate code, which was originally written in 
Mathematica and described in 
\href{http://arxiv.org/abs/1308.6288v1}{arXiv:1308.6288}.  We call our program
{\tt DMFortFactor}, a Fortran interpretation of DMFormFactor.

This program is concerned principally with computing the WIMP-nucleus differential 
event rate as a function of the nuclear recoil energy $E_R$:
\begin{dmath}
\frac{dR_D}{dE_R} = \frac{dR_D}{d\vec{q}^2}(q)
	\\= N_T\frac{\rho_\chi}{m_\chi}\int_{v_{min}}^{v_{escape}} 
	\frac{2m_T}{4\pi v^2}\frac{1}{2j_\chi+1}\frac{1}{2j_T+1}
	\sum_{spins}|\mathcal{M}(v,q)|^2  \tilde{f}(\vec{v})vd^3v
\end{dmath}
This quantity has units of events/GeV and is implicitly multiplied by
an effective exposure of 1 Kilogram-Day of target nuclei. This is done by
taking $N_t = 1\ kilogram\cdot day / m_T$, where $m_T$ is the mass of the target
nucleus in GeV. Recoil energies $E_R$ are given in keV.

The cross section is determined using a user-defined WIMP-nucleus interaction
within a non-relativistic effective field theory (EFT) framework. The
interaction is specified by 16 coupling coefficients defining an interaction:
\begin{equation}
	\sum_{x=p,n}\sum_{i=1}^{16} c_i^x \mathcal{O}_i.
\end{equation}

\section{Quickstart guide}
{\tt DMFortFactor} can be used interactively from the command line,  where the
user is guided by prompts for a small number of datafiles and parameters.
Naturally, this interactive process can be expedited by piping a pre-written
input file into the command line interface (CLI).

We also provide a generic application programming interface (API) for the Python
language.  This API essentially offers a prescribed and easy-to-use way to run
{\tt DMFortFactor} in a programmatic way.  Any sufficiently advanced linux user
could probably write a script to produce any possible set of inputs to our code.
But our API removes the need by making it easy for anyone who can use a Python
function to write their own advanced scripts allowing them to perform parameter
studies and comparisons of different inputs to the theory.

\subsection{SuperQuickstart guide}
\begin{enumerate}
	\item Navigate to the {\tt src/} directory from wherever you have stored {\tt darkmatter/} (e.g. {\tt cd src/}, or {\tt cd $\sim$/Downloads/darkmatter/src/})
	\item Run the command: {\tt make new}
	\item Navigate to the directory {\tt runs/xe/} (e.g. {\tt cd ../runs/xe/})
	\item Run the command: {\tt ../../src/dmfortfactor.x < xe131.input}
\end{enumerate}
If successful,  one of the last lines printed to screen should be ``Event rate
spectra written to eventrate_spectra.dat''.

\subsection{Compiling with make}
There are multiple directories in the project. All of the Fortran code which
needs to be compiled is found in the {\tt src/} directory.  We provide a {\tt
Makefile} to compile the code using a few different options. 

The easiest way to get started is simply to navigate to {\tt src/}  and run
\begin{lstlisting}
make dmfortfactor
\end{lstlisting}
This will compile {\tt DMFortFactor} using {\tt gfortran}.  If you want to use a
different compiler, you must edit the Makefile to change the {\tt COMP} variable
from gfortran to your compiler of choice.

Aditionally, if you want to utilize multiple CPU cores to speed up the runtime,
you can compile with {\tt openMP}: 
\begin{lstlisting}
make openmp
\end{lstlisting}
Both of these options will compile the source code and leave the executable,
called {\tt dmfortfactor.x} in the {\tt src/} directory.  Note that if you
change from a serial executable to a parallel executable (or vice versa) you
should run:
\begin{lstlisting}
make clean
\end{lstlisting}
in between builds.

\subsection{Required files}
There are two files required for any calculation:
\begin{enumerate}
    \item Control file (.control)
    \item Nuclear density matrix file (.dres)
\end{enumerate}
Additionally, if the user enables the option ``usenergyfile'', then a file
containing the input energies or momentum will also be required.

If you use the {\tt EventrateSpectra} Python wrapper, then the control file will
automatically be generated.

\subsubsection{Control file (.control)}
Each EFT parameter is written on its own line in [mycontrolfile].control, with
four values: the keyword "coefnonrel", the operator number (integer 1..16), the
coupling type ("p"=proton, "n"=neutron, "s"=scalar, "v"=vecctor), and the
coefficient value. For example, 
\begin{verbatim}
coefnonrel    1    s     3.1
\end{verbatim}
would set $c_1^0 = 3.1$. We take the isospin convention:
\begin{equation}
	\begin{split}
		c^0 = c^p + c^n\\
		c^1 = c^p - c^n
	\end{split}
\end{equation}
Thus, the previous example is equivalent to:
\begin{verbatim}
coefnonrel    1    p     1.55
coefnonrel    1    n     1.55
\end{verbatim}

The control file also serves a more general but optional function: to set any
parameter in the program to a custom value.  Simply add an entry to the control
file with two values: the first should be the keyword for the parameter and the
second should be the value to set that parameter to. For example, to set the
velocity of the earth in the galactic frame to $240\ km/s$, you should add the
line:
\begin{verbatim}
vearth  240.0
\end{verbatim}
A complete list of keywords is given in section \ref{cfk}.

\subsubsection{Nuclear density matrix file (.dres)}

\subsection{Event rate spectra (events per GeV) using the CLI}
The program will prompt the user for the minimum necessary inputs to run a
calculation with default parameter values, including the name of a control file
which contains the EFT coefficients, and optionally, additional customizations
to the calculation parameters.

After selecting the option to compute an event rate spectra, there are six
further lines of input. These will be explained by an example:

\begin{verbatim}
 Enter the target proton number
54
 Enter the target neutron number 
77 
 Enter name of control file (.control):
xe131
...
  Enter name of one-body density file (.dres)
xe131gcn
...
 What is the range of recoil energies in kev?
 Enter starting energy, stoping energy, step size:
0.0001 250. 1.0
\end{verbatim}

The first two entries are self-evident: we specifiy the number of protons and
neutrons in the target nucleus. In this case, 54 and 74, respectively, for
$^{131}$Xe.

Third is the name of the control file containing the EFT coefficients and other,
optional, settings. The `.control' file extension should be omitted. This 
contents of this file will be explained in more detail later.

Fourth is the file containing the nuclear wave functions in the form of one-body
density matrices. Only the ground-state need be included. The '.dres' file
extension is omitted. 

Fifth and finally are three numbers specifying the range of recoil energies
$E_R$ that the differential scattering rate should be computed for.

The event rate spectra will be written to a file, and as a side effect of the
calculation, the total event rate for the energy range requested will be
estimatted by numerical integration. Note that the accuracy of this result will
depend on the choice of the step size.

\subsection{Event rate spectra from the Python wrapper}
We provide a Python wrapper for the code and a number of example scripts
demonstrating its use. The wrapper comes in the form of a Python function {\tt
EventrateSpectra} which can be imported from dmfortfactor.py in the Python
directory.  This function has three required arguments: 
\begin{enumerate}
    \item Z the number of protons in the target nucleus
    \item N the number of neutrons in the target nucleus
    \item The `.dres' filename for the one-body density matrix file describing the nuclear structure. 
\end{enumerate}
If no other arguments are provided, default values will be used for all of the
remaining necessary parameters, including zero interaction strength.  To
calculate an event rate with a nonzero interaction, the user should also provide
one or more of the optional EFT coupling coefficient arrays: {\tt cpvec, cnvec,
csvec, cvvec}. These store the couplings to protons, neutrons, isoscalar, and
isovector, respectively.  Finally, the user can also pass a dictionary of valid
control keywords and values to the function in order to set any of the control
words defined in the manual.

To compute the eventrate spectra for $^{131}$Xe with a WIMP mass of 50 GeV and a
$c_3^v=0.0048$ coupling, one might call:
\begin{lstlisting}[language=Python]
import dmfortfactor as dm
control_dict = {"wimpmass" : 50.0}
cvvec = np.zeros(15)
cvvec[2] = 0.0048 
Erkev, ER = dm.EventrateSpectra(
            Z = 54,
            N = 77,
            dres = "xe131gcn",
            controlwords = control_dict,
            cvvec = cvvec)
\end{lstlisting}
This will return the differential event rate spectra for recoil energies from 1
keV to 1 MeV in 1 keV steps. The file `xe131gcn.dres' must exist in the current
working directory and contain a valid one-body density matrix for $^{131}$Xe.


\section{Python interface}

We provide two generic API's for interacting with the Fortran program. They are:
{\tt runTemplates} and {\tt EventrateSpectra}.  These can be imported into your
own Python script by having the file {\tt dmfortfactoy.py} in your working
directory or by adding it to your path:
\begin{lstlisting}[language=Python]
import sys
sys.path.append('/path/to/dmfortfactor.py')
import dmfortfactor as dm
\end{lstlisting}

{\tt EventrateSpectra} is essentially a wrapper for the event-rate calculation
function of {\tt dmfortfactor}. It makes use of {\tt runTemplates} but shields
the user from having to handle the input and control files needed by {\tt
dmfortfactor}.

{\tt runTemplates} is a fairly generic function which runs an arbitrary linux
program which takes an input file by a linux pipe, and makes use of a secondary
`control' file. {\tt runTemplates} takes in templates for the input and control
files, modifies them using {\tt sed} to replace string keys with values from a
Python dictionary, runs the executable, then collects outputs written to an
output file, returning the data as Numpy arrays.

\subsection{Example: generating a weighted sum of event rate spectra}

\begin{lstlisting}[language=python]
import sys
sys.path.append("/Users/oliver/projects/darkmatter/python")
import numpy as np
import matplotlib.pyplot as plt
import dmfortfactor as dm

plt.figure()
Z = 54
isotopes = [128, 129, 130, 131, 132, 134, 136]
weights = [.01910, .26401, .04071, .21232, .26909, .10436, .08857]
weightedsum = 0.0

for i,isotope in enumerate(isotopes):
    N = isotope - Z
    controls = {
            "wimpmass" :500.0,
            "vescape" : 550.0}
    cp = np.zeros(15)
    cp[0] = 1
    RecoilE, EventRate = dm.EventrateSpectra(
            Z,
            N,
            dres = "xe%igcn"%isotope,
            controlwords = controls,
            cpvec = cp )

    weightedsum += EventRate * weights[i]

    label = "xe"+str(isotope)
    plt.plot(RecoilE, EventRate, label="$^{%s}$Xe"%isotope)
\end{lstlisting}
We now have the weighted sum of event rates stored in the variable weightedsum.
Using matplotlib, we could produce the following plot:

{
	\centering
\includegraphics[width=\textwidth]{weightedspectra.pdf}

}

\subsection{Example: comparing event-rate spectra for
calculations with different WIMP masses}


\clearpage

\section{Other compute options}
There are a handful of other options available from the main menu of the code.  Presently, the primary use-cases for these options are validation and debugging.  the compute-options are:
\begin{itemize}
	\item [{[2]}] Differential cross section at a fixed recoil energy over a range of speeds
	\item [{[3]}] Transition/scattering probability at a fixed recoil energy over a range of speeds
	\item [{[5]}] Total integrated events (without producing spectra file)
	\item [{[6]}] Nuclear response functions at a given value of $y=(qb/2)^2$
\end{itemize}
The number is square brackets [x] will be refered to as the compute-option.
Once a compute-option is chosen,  subsequent CLI inputs are the same up until the density matrix file (.dres) has been read-in. Then, the inputs depend on the compute-option chosen.
\begin{itemize}
	\item [{[2]}] Differential cross section per recoil energy. Four additional inputs:
		\begin{itemize}
			\item E-recoil (keV)
			\item v-start (km/s)
			\item v-stop (km/s)
			\item v-step (km/s)
		\end{itemize}
	\item [{[3]}] Scattering probability. Same as [2].
	\item [{[5]}] Total scattering events per detector (does not produce spectra data).  This option uses adaptive quadrature to perform the integral of the event rate spectra with the fewest number of evaluations to reach the desired relative error.  This will be much faster than the result from options [1]. Three additional inputs:
		\begin{itemize}
			\item E-start (keV)
			\item E-stop (keV)
			\item Desired relative error (decimal value)
		\end{itemize}
	\item [{[6]}] Nuclear response function test. This compute-option allows the user to evaluate the nuclear response functions $W_i^{x,x'}(y)$ for a provided value of $y$. All combinations of $x$ and $x'$ will be printed for both isospin and proton-neutron couplings.  Two additional inputs are required:
	\begin{itemize}
		\item Function number (1 - 8)
		\item Value of $y=(qb/2)^2$ (dimensionless)
	\end{itemize}
\end{itemize}


\section{Equations found in the code}
We present the equations necessary to reproduce the code. For a more complete
description of the theory, see 
\href{https://link.aps.org/doi/10.1103/PhysRevC.89.065501}{Phys. Rev. C 89.065501.}
\subsection{Differential event rate}
\begin{equation}\label{ER}
\begin{split}
	\frac{dR_D}{dE_R} = \frac{dR_D}{d\vec{q}^2}(q)
	 = N_T n_\chi \int_{v_{min}}^\infty \frac{d\sigma(v,q)}{d\vec{q}^2} \tilde{f}(\vec{v})vd^3v
\end{split}
\end{equation}
where $q$ is the WIMP-nucleon momentum transfer, $N_T$ is the number of target 
nuclei, $n_\chi = \rho_\chi/m_\chi$ is the local dark matter number density, $\sigma$ 
is the WIMP-nucleon cross section, and $\tilde{f}$ is dark matter velocity 
distribution in the lab-frame. $\tilde{f}(\vec{v})$ is obtained by boosting 
the Galactic-frame distribution $f(\vec{v})$, 

\begin{equation}\label{boost}
\begin{split}
	\tilde{f}(\vec{v}) = f(\vec{v} + \vec{v}_{earth}),
\end{split}
\end{equation}
where $\vec{v}_{earth}$ is the velocity of the earth in the galactic rest 
frame. The simplest model is a three-dimensional Maxwell distribution:

\begin{equation}
\begin{split}
	f(\vec{v}) \propto e^{-\vec{v}^2/v_0^2},
\end{split}
\end{equation}
where $v_0$ is some scaling factor (typically taken to be around $220\ km/s$).

In order to evaluate the integral in (\ref{ER}), we make the conversion to 
spherical coordinates, and take special care to deal with the velocity boost 
in (\ref{boost}). Assuming a $1/v^2$ velocity dependence of the cross section 
term (see section \ref{crosssection}), we need to evaluate an integral of the 
form
\begin{equation}
\begin{split}
	I = \int_{v_{min}}^{v_{max}} d^3v \frac{f(\vec{v} + \vec{v}_{earth})}{v} =
    	\int_{v_{min}}^{v_{max}} d^3v \frac{1}{v} e^{-(\vec{v}+\vec{v}_{earth})^2/v_0^2}
\end{split}
\end{equation}
Noting that $(\vec{v}+\vec{v}_{earth})^2 = \vec{v}^2 + \vec{v}^2_{earth} + 
2vv_{earth}\cos(\theta)$, with $|\vec{v}|\equiv v$ and $\theta$ defining 
the angle between the two vectors, it's convenient to make the substitution 
$d^3v = d\phi d(\cos \theta) v^2 dv$:
\begin{equation}
\begin{split}
	I &=  \int_0^{2\pi} d\phi \int_{v_{min}}^{v_{max}} dv \int_{-1}^1 d(\cos \theta) e^{-2vv_{earth}\cos\theta/v_0^2} v^2 \frac{1}{v} e^{-(\vec{v}^2+\vec{v}^2_{earth})/v_0^2}\\
	&= 2\pi \int_{v_{min}}^{v_{max}} dv v e^{-(\vec{v}^2+\vec{v}^2_{earth})/v_0^2} \left(-\frac{v_0^2}{2vv_{earth}} e^{-2vv_{earth}\cos\theta/v_0^2}\right)_{-1}^1\\
	&= \frac{\pi v_0^2}{v_{earth} }\int_{v_{min}}^{v_{max}} dv e^{-(\vec{v}^2+\vec{v}^2_{earth})/v_0^2} 
		\left(- e^{-2vv_{earth}/v_0^2} + e^{+2vv_{earth}/v_0^2}\right)\\
	&= \frac{\pi v_0^2}{v_{earth} }\int_{v_{min}}^{v_{max}} dv 
		\left(- e^{(v+v_{earth})^2/v_0^2} + e^{(v-v_{earth})^2/v_0^2}\right)\\
	&= \frac{\pi v_0^2}{v_{earth} }\int_{v_{min}}^{v_{max}} dv 
		\left( g(v-v_{earth}) - g(v+v_{earth}) \right)
\end{split}
\end{equation}
where in the last equality, we have defined a one-dimensional Gaussian form
\begin{equation}
\begin{split}
	g(v) \propto e^{-v^2/v_0^2}.
\end{split}
\end{equation}

The final expression for $I$ can be trivially generalized to other spherically
symmetric velocity-dependent forms of the differential cross section. What's 
important is the reduction of the velocity-boosted $d^3v$ integral to a radial 
integral which can be carried out with one-dimensional quadrature:
\begin{equation}\label{integral}
\begin{split}
\int_{v_{min}}^{v_{max}} d^3v \sigma(v) e^{-(\vec{v}+\vec{v}_{earth})^2/v_0^2}
\\
	= \frac{\pi v_0^2}{v_{earth} }\int_{v_{min}}^{v_{max}} dv \sigma(v) v^2\left( g(v-v_{earth}) - g(v+v_{earth}) \right).
\end{split}
\end{equation}
The FORTRAN code uses equation (\ref{integral}) to evaluate the event rate 
integral in equation (\ref{ER}) with quadrature. Analytic solutions of  
(\ref{integral}) exist in the form of error functions; we use the above form 
since it makes easy to later modify the velocity distribution (as long as it 
remains spherically symmetric). For example, adding a velocity cut-off is as 
easy as changing the limit on the quadrature, with no need to write a whole 
new subroutine for the analytic forms found in the Mathematica script.

\subsection{Differential cross section}\label{crosssection}
\begin{equation}
\begin{split}
\frac{d\sigma(v,E_R)}{dE_R} = 2m_T \frac{d\sigma(v)(v,\vec{q}^2)}{d\vec{q}^2} = 2m_T\frac{1}{4\pi v^2}T(v,q),
\end{split}
\end{equation}
Where $v$ is the velocity of the dark matter particles in the lab-frame, $q$ 
is the momentum transfer of the scattering event, $m_T$ is the mass of the 
target nucleus, and $T(v,q)$ is the transition or scattering probability. We 
can see here that the differential cross section has an explicit $1/v^2$ 
dependence, independent of any velocity dependence of $T(v,q)$.


\subsection{Transition probability / Scattering probability}

The scattering probability is

\begin{equation}
\begin{split}
T(v,q) &= \frac{1}{2j_\chi+1}\frac{1}{2j_T+1}\sum_{spins}|\mathcal{M}(v,q)|^2 
\end{split}
\end{equation}
where $j_\chi$ is the spin of the WIMP, $j_T$ is the spin angular momentum of 
the target nucleus, and $\mathcal{M}$ Galilean invariant amplitude, which is 
defined by
\begin{align}
	T(v,q) = \frac{4\pi}{2j_T+1}\frac{1}{(4m_\chi)^2}
		\sum_{x=p,n}\sum_{x'=p,n}^1\sum_{i=1}^8 R_i^{xx'}(v^2,q^2)
		W_i^{xx'}(q)
%		\left < \mathcal{O}_{j_T,x}^{(i)}(y)\right >
%		\left < \mathcal{O'}_{j_T,x'}^{(i)}(y)\right >
\end{align}
where $m_\chi$ is the mass of the dark matter particle and $x$ is an index used 
to sum over isospin couplings. The coefficients $R_i^{x,x'}$ are dark matter 
particle response functions, to be define in another section. The operators 
$W_i^{xx'}(q)$ are nuclear response functions, which are sums over matrix 
elements of nuclear operators constructed from Bessel spherical harmonics and 
vector spherical harmonics.
\subsection{Dark matter response functions}
There are 8 dark matter response functions which group 15 operator coefficients
$c_i^x$ according the pair of nuclear response functions which they multiply.

%%%%%%%%%%%%%%%%%%%%%%%%%%%%%%%%%%%%%%%%%%%%%%%%%%%%%%%%%%%%%%%%%%%%%%%%%%%%%%80
% Tex formatting style for this section: 
%   * parenthesis hierarchy: \{, [, (
%   * zero indent for new definition or equality
%   * single indent for new print line (usage of \\)
%   * double indent for source line continuation without new print line
\begin{dmath}
R_{M}^{xx'}(v,q) = \frac{1}{4}cl(j_\chi) \{ [v^2-(q/2\mu_t)^2] 
        (c_5^{x}c_5^{x'}q^2 + c_8^{x}c_8^{x'}) + c_{11}^{x}c_{11}^{x'}q^2 \}
    + \{c_1^{x} + c_2^{x}[v^2-(q/2\mu_t)^2] \} \{c_1^{x'} 
        + c_2^{x'}[v^2-(q/2\mu_t)^2] \} 
\end{dmath}
%%%%%%%%%%%%%%%%%%%%%%%%%%%%%%%%%%%%%%%%%%%%%%%%%%%%%%%%%%%%%%%%%%%%%%%%%%%%%%80
\begin{dmath}
R_{\Sigma''}^{xx'}(v,q) = \frac{1}{16}cl(j_\chi) \{c_6^{x}c_6^{x'}q^4 
    + (c_{13}^{x}c_{13}^{x'}q^2 + c_{12}^{x} c_{12}^{x'} ) [v^2 
    - (q/2\mu_T)^2 ] + 2c_4^xc_6^{x'}q^2 + c_4^xc_4^{x'}\} 
    + \frac{1}{4}c_{10}^xc_{10}^{x '}q^2
\end{dmath}
%%%%%%%%%%%%%%%%%%%%%%%%%%%%%%%%%%%%%%%%%%%%%%%%%%%%%%%%%%%%%%%%%%%%%%%%%%%%%%80
\begin{dmath}
R_{\Sigma'}^{xx'}(v,q) = \frac{1}{32} cl(j_\chi) \{ 2c_{9}^{x}c_{9}^{x'}q^2 
        + ( c_{15}^{x}c_{15}^{x'}q^4 + c_{14}^{x}c_{14}^{x'}q^2 
        - 2c_{12}^{x}c_{15}^{x'} q^2 + c_{12}^{x}c_{12}^{x'}) [v^2-(q/2\mu_T)^2]
        + 2c_{4}^{x}c_{4}^{x'} \} 
        +\frac{1}{8}(c_{3}^{x}c_3^{x'}q^2 + c_{7}^{x}c_{7}^{x'})[v^2-(q/2\mu_T)^2]
\end{dmath}
%%%%%%%%%%%%%%%%%%%%%%%%%%%%%%%%%%%%%%%%%%%%%%%%%%%%%%%%%%%%%%%%%%%%%%%%%%%%%%80
\begin{dmath}
R_{\Phi''}^{xx'}(v,q) = \frac{q^2}{(4m_N)^2}cl(j_\chi) (c_{12}^x - c_{15}^{x}q^2
        )(c_{12}^{x '}-c_{15}^{x '}q^2 ) 
    + \frac{q^2}{(4m_N)^2}q^2c_3^x c_3^{x'} 
\end{dmath}
%%%%%%%%%%%%%%%%%%%%%%%%%%%%%%%%%%%%%%%%%%%%%%%%%%%%%%%%%%%%%%%%%%%%%%%%%%%%%%80
\begin{dmath}
R_{\tilde{\Phi}'}^{xx'}(v,q) = \frac{q^2}{(4m_N)^2}cl(j_\chi)(
        c_{12}^xc_{12}^{x'}q^2 + c_{12}^x c_{12}^{x'})
\end{dmath}
%%%%%%%%%%%%%%%%%%%%%%%%%%%%%%%%%%%%%%%%%%%%%%%%%%%%%%%%%%%%%%%%%%%%%%%%%%%%%%80
\begin{dmath}
R_{\Delta}^{xx'}(v,q) = \frac{q^2}{(2m_N)^2}cl(j_\chi) (c_{5}^{x}c_{5}^{x'}q^2 
        + c_{8}^{x}c_{8}^{x'}) 
        + 2\frac{q^2}{m_N^2}c_{2}^{x}c_{2}^{x'}[v^2-(q/2\mu_T)^2]
\end{dmath}
%%%%%%%%%%%%%%%%%%%%%%%%%%%%%%%%%%%%%%%%%%%%%%%%%%%%%%%%%%%%%%%%%%%%%%%%%%%%%%80
\begin{dmath}
R_{\Delta \Sigma'}^{xx'}(v,q) = \frac{q^2}{(2m_N)^2}cl(j_\chi) 
        (c_{4}^{x}c_{5}^{x'} - c_{8}^{x}c_{9}^{x'}) 
        - \frac{q^2}{m_N} c_{2}^{x}c_{3}^{x'} [v^2-(q/2\mu_T)^2]
\end{dmath}
%%%%%%%%%%%%%%%%%%%%%%%%%%%%%%%%%%%%%%%%%%%%%%%%%%%%%%%%%%%%%%%%%%%%%%%%%%%%%%80
\begin{dmath}
R_{\Phi''M}^{xx'}(v,q) = \frac{q^2}{4m_N}cl(j_\chi)c_{11}^{x}
        (c_{12}^{x'} - c_{15}^{x'} q^2) 
        + \frac{q^2}{m_N}c_{3}^{x'}  \{c_{1}^{x} + c_{2}^{x} [v^2-(q/2\mu_T)^2]\}
\end{dmath}
%%%%%%%%%%%%%%%%%%%%%%%%%%%%%%%%%%%%%%%%%%%%%%%%%%%%%%%%%%%%%%%%%%%%%%%%%%%%%%80
As a shorthand we have introduced the notation 
\begin{align}
	cl(j) = 4j(j+1)/3.
\end{align}


\subsection{Nuclear response functions}
There are eight nuclear response functions $W_i^{xx'}(y)$ considered 
here. The unit-less variable $y$ is defined 
\begin{align}
 y = \left ( \frac{qb}{2} \right) ^2,
 \end{align}
 in terms of the harmonic oscillator size parameter $b$, which has a default value of 
 \begin{align}\label{bho}
b^2 = 41.467/(45A^{-1./3} - 25A^{-2/3})\ fm^2.
 \end{align}

 
\begin{align}
W_M^{xx'}(y) &= \sum_{even\ J} 
\bra{j_T}M_{Jx}(y)\ket{j_T}
\bra{j_T}M_{Jx'}(y)\ket{j_T}
\\
W_{\Sigma''}^{xx'}(y) &= \sum_{odd\ J} 
\bra{j_T}{\Sigma''}_{Jx}(y)\ket{j_T}
\bra{j_T}{\Sigma''}_{Jx'}(y)\ket{j_T}
\\
W_{\Sigma'}^{xx'}(y) &= \sum_{odd\ J} 
\bra{j_T}{\Sigma'}_{Jx}(y)\ket{j_T}
\bra{j_T}{\Sigma'}_{Jx'}(y)\ket{j_T}
\\
W_{\Phi''}^{xx'}(y) &= \sum_{even\ J} 
\bra{j_T}{\Phi''}_{Jx}(y)\ket{j_T}
\bra{j_T}{\Phi''}_{Jx'}(y)\ket{j_T}
\\
W_{\tilde{\Phi}'}^{xx'}(y) &= \sum_{even\ J} 
\bra{j_T}{\tilde{\Phi}'}_{Jx}(y)\ket{j_T}
\bra{j_T}{\tilde{\Phi}'}_{Jx'}(y)\ket{j_T}
\\
W_{\Delta}^{xx'}(y) &= \sum_{odd\ J} 
\bra{j_T}{\Delta}_{Jx}(y)\ket{j_T}
\bra{j_T}{\Delta}_{Jx'}(y)\ket{j_T}
\\
W_{\Delta\Sigma'}^{xx'}(y) &= \sum_{odd\ J} 
\bra{j_T}{\Delta}_{Jx}(y)\ket{j_T}
\bra{j_T}{\Sigma'}_{Jx'}(y)\ket{j_T}
\\
W_{\Phi''M}^{xx'}(y) &= \sum_{even\ J} 
\bra{j_T}{\Phi''}_{Jx}(y)\ket{j_T}
\bra{j_T}{M}_{Jx'}(y)\ket{j_T}
\end{align}


\subsection{Nuclear operators and their matrix elements}
There are six nuclear operators constructed from Bessel 
spherical harmonics and vector spherical harmonics, and are evaluated here on 
the ground state of the target nucleus.

There are six nuclear operators $\mathcal{W}^{(f)}$, $f=1,...,6$, describing the electro-weak coupling of the WIMPs to the nucleon degrees of freedom. The six single-particle operators are given the symbols:
\begin{equation}
    \mathcal{W}^{(f=1,...,6)}_J=M_J, \Delta_J, \Sigma_J', \Sigma_J'', \tilde{\Phi}_J', \Phi_J'',
\end{equation}
and are constructed from Bessel spherical and vector harmonics \cite{DONNELLY1979103}:
\begin{dmath}
    M_{JM}(q\vec{x})\equiv j_J(qx)Y_{JM}(\Omega_x)
\end{dmath}
\begin{dmath}
    \vec{M}_{JML}(q\vec{x}) \equiv j_L(qx) \vec{Y}_{JLM}(\Omega_x),
\end{dmath}
where
\begin{dmath}
    Y_{JLM}(\Omega_x) = \sum_{m\lambda} \bra{Lm1\lambda}\ket{(L1)JM_J} Y_{Lm}(\Omega_x)\vec{e}_\lambda.
\end{dmath}
The six multipole operators are defined as:
\begin{align}
    \label{oplist}
M_{JM}\ \ &\\
\Delta_{JM} \equiv& \vec{M}_{JJM}\cdot \frac{1}{q}\vec{\nabla}\\
\Sigma'_{JM} \equiv& -i \left \{\frac{1}{q}\vec{\nabla}\times \vec{M}_{JJM}  \right\}\cdot \vec{\sigma}\\
\Sigma''_{JM} \equiv& \left \{ \frac{1}{q}\vec{\nabla}M_{JM} \right \}\cdot \vec{\sigma}\\
\tilde{\Phi}'_{JM} \equiv& \left( \frac{1}{q} \vec{\nabla} \times \vec{M}_{JJM}\right)\cdot \left(\vec{\sigma}\times \frac{1}{q}\vec{\nabla} \right) + \frac{1}{2}\vec{M}_{JJM}\cdot \vec{\sigma}\\
\Phi''_{JM}\equiv& i\left(\frac{1}{q}\vec{\nabla}M_{JM} \right)\cdot \left(\vec{\sigma}\times \frac{1}{q}\vec{\nabla} \right)
\end{align}

The matrix elements of these operators can be calculated for standard wave functions from second-quantized shell model calculations:
\begin{equation*}
    \bra{\Psi_f} \mathcal{W}^{(f)}_J \ket{\Psi_i} = \Tr(\mathcal{W}^{(f)}_J \rho^{f,i}_J )
\end{equation*}
\begin{equation}
 = \sum_{a,b}
     \bra{a} |\mathcal{W}^{(f)}_J| \ket{b}
     \rho^{fi}_J(ab),
\end{equation}
where single-particle orbital labels $a$ imply shell model quantum number $n_a, l_a, j_a$,
and the double-bar $||$ indicates reduced matrix elements~\cite{edmonds1996angular}.
We assume a harmonic oscillator single-particle basis, with the important convention that the
radial nodal quantum number $n_a$ starts at 0, that is, we label the orbitals as $0s, 0p, 1s0d$, etc..,
and \textit{not} starting with $1s, 1p,$ etc.
Then the one-body matrix elements for operators $ \bra{a} |\mathcal{W}^{(f)}_J| \ket{b}$, built from spherical Bessel functions and vector spherical harmonics,  have closed-form expressions in terms of confluent hypergeometric
functions~\cite{DONNELLY1979103}.

The matrix element of the Bessel function on harmonic oscillator wave functions is
\begin{equation}
    \begin{split}
    \bra{1l'}j_L(y)\ket{1l} = 
        \frac{
            (2y)^{L/2}e^{-y}(L+l'+l+1)!!
        }{
            (2L+1)!![(2l'+1)!!(2l+1)!!]^{1/2}
        }\\
        \times\ _1F_1((L-l'-l)/2, L+\frac{3}{2}, y),
    \end{split}
\end{equation}    
where $_1F_1$ is the confluent hypergeometric function:
\begin{equation}
    _1F_1(a,b,z) = \sum_{n=0}^\infty \frac{a^{(n)}z^n}{b^{(n)}n!},
\end{equation}
which makes use of the rising factorial function:
\begin{equation}
    m^{(n)} = \frac{(m+n-1)!}{(m-1)!}.
\end{equation}

\section{Control file keywords}\label{cfk}

\begin{longtable}{| l | c | p{2.5in} | c | l | }
  \hline Keyword & Symbol & Meaning & Units & Default \\

  \hline dmdens & $\rho_\chi$ & Local dark matter density. & GeV/cm$^3$ & 0.3\\

  \hline dmspin & $j_\chi$ & Instrinsic spin of WIMP particles. & $\hbar$ &
  $\frac{1}{2}$ \\

  \hline fillnuclearcore & & Logical flag (enter 0 for False, 1 for True) to fill the
  inert-core single-particle orbitals in the nuclear level densities.
  Phenomenological shell model calculations typically provide only the density
  matrices for the active valence-space orbitals, leaving it to the user to
  infer the core-orbital densities. This option automatically assigns these
  empty matrix elements assuming a totally filled core. & & 1 (true)\\
  
  \hline gaussorder &  & Order of the Gauss-Legendre quadrature to use when using Type 2 quadrature. (See quadtype.) An n-th order routine will perform n function evaluations.  Naturally, a higher order will result in higher precision, but longer compute time. & & 12\\

  \hline hofrequency & $\hbar \omega$ & Set the harmonic oscillator length by
  specifying the harmonic oscillator frequency. (b = 6.43/sqrt($\hbar\omega$)).
  If using an \textit{ab initio} interaction, $\hbar \omega$ should be set to
  match the value used in the interaction.
              & MeV & See hoparameter.\\

  \hline hoparameter & $b$ & Harmonic oscillator length. Determines the scale of the
  nuclear wavefunction interaction. & fm &
  See eqn. (\ref{bho}).  \\

  \hline maxwellv0 & $v_0$ & Maxwell-Boltzman velocity distribution scaling factor. &
  km/s & 220.0 \\

  \hline mnucleon & $m_N$ & Mass of a nucleon. It's assumed that $m_p\approx m_n$. &
  GeV & 0.938272 \\

  \hline ntscale & $N_t$ & Effective number of target nuclei scaling factor. The
  differential event rate is multiplied by this constant in units of
  kilogram-days. For example, if the detector had a total effective exposure of
  2500 kg days, one might enter 2500 for this value. & kg days & 1.0 \\
  
  \hline quadrelerr &  & Desired relative error for the adaptive numerical quadrature routine (quadtype 1).  & & $10^{-6}$\\
  
  \hline quadtype & & Option for type of numerical quadrature. (Type 1 = adaptive 8th order Gauss-Legendre quadrature.  Type 2 = static n-th order Gauss-Legendre quadrature.) && 1 (type 1)\\

  \hline sj2tablemax & & Maximum value of $2\times J$ used when caching Wigner 3-J and 6-J functions into memory. & & 12\\
  
  \hline sj2tablemin & & Minimum value of $2\times J$ used when caching Wigner 3-J and 6-J functions into memory. & & -2\\
  

  \hline useenergyfile & & Logical flag (enter 0 for False, 1 for True) to read energy
  grid used for calculation from a user-provided file intead of specifying a
  range. & & 0 (false) \\

  \hline usemomentum & & Logical flag (enter 0 for False, 1 for True) to use momentum
  transfer intead of recoil energy as the independent variable. & &0 (false) \\

  \hline vearth & $v_{earth}$ & Speed of the earth in the galactic frame. & km/s & 
  232.0\\

  \hline vescape & $v_{escape}$ & Galactic escape velocity. Particles moving faster than
  this speed will escape the galaxy, thus setting an upper limit on the WIMP
  velocity distribution. & km/s & 12 $\times\ v_{scale}$ \\

  \hline weakmscale & $m_v$ & Weak interaction mass scale. User defined EFT coefficients
  are divided by $m_v^2$. & GeV & 246.2 \\

  \hline wimpmass & $m_\chi$ & WIMP particle mass. & GeV & 50.0\\
  \hline
\end{longtable}

\end{document}
